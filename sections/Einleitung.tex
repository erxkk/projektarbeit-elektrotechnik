\subsection{Begrifflichkeiten}
    \subsubsection{Photovoelektrischer Effekt}
        Wechselwirkung von Photonen mit baryonischer Materie, getrennt
        in inneren und äußeren photovoelektrischen Effekt und die
        Photoionisation. Beschreibt die Freisetzung von Elektronen durch
        Bestrahlung eines Materials mit elektromagnetischer Strahlung
        \cite{Wiki_PhotoelectricEffect}.

    \subsubsection{Photovoltaischer Effekt}
        Teil des inneren photoelektrischen Effekts, beschreibt Bildung
        eines Photostroms, also Trennung von Ladungsträgerpaaren in den
        p- und n-dottierten Schichten einer Photodiode entgegen der
        Durchlassrichtung des Übergangs als Folge von elektromagnetischer
        Strahlung auf eine Photodiode. Der photovolatische Effekt baut auf
        der Photoleitung auf, einem weitern innerem Teil des photoelektrischen
        Effekts \cite{Wiki_PhotoelectricEffect}.\\
        Der photovoltaische Effekt dient als Grundlage für die Funktionsweise
        von photovoltaischen Zellen.

    \subsubsection{Photovoltaische Zelle}
        Elektrisches Bauelement das auf Grundlage des photvoltaischen
        Effekts, Strom erzeugt und aus Halbleitermaterialien (vorwiegend
        Silizium) besteht.\\
        Umganssprachlich werden photovoltaische Zellen oft Solarzellen
        genannt, dabei wird der Begriff oft als Synonym für Solarpanäle
        benutzt obwohl diese eine Ansammlung von photolotaischen Zellen,
        Leiterelementen und strukturellen Bauelementen beschreiben.

\subsection{Historie der Photovoltaik}
    \subsubsection{Entdeckung}
        Die Effekte der Photovoltaik wurden erstmals in 1839 von Andre Edmond
        Becquerel entdeckt, aber erst weit später praktisch angewendet.
        \cite{Wiki_PhotovoltaicHistory}

    \subsubsection{Nennenswerte Ereignisse}
        \begin{itemize}
            \item 1876 - Beweis der direkten Konversion von
                elektromagnetischer Strahlung in elektrische Energie durch
                William Grylls Adams und Richard Evans Day
            \item 1907 - Theoretische Erklärung des photoelektrischen Effekts
                auf Basis der Lichtquantenhypothese (1905) durch Albert
                Einstein
            \item 1912 - 1916 - Experimentelle Bestätigung von Einsteins
                Erklärung durch Robert Adndrews Millikan
            \item 1958 - Erste Verwendung von Solarzellen zur Versorgung
                eines Satelliten der NASA (Vanguard I) \cite{Wiki_Vanguard}
        \end{itemize}