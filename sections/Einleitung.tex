\subsection{Begrifflichkeiten}
    \subsubsection{Photoelektrischer Effekt}
        Wechselwirkung zwischen Photonen und baryonischer Materie,
        wird in den inneren und äußeren photovoelektrischen Effekt
        und die Photoionisation unterteilt. Beschreibt die Freisetzung
        von Elektronen durch Bestrahlung eines Materials mit
        elektromagnetischer Strahlung \cite{Wiki_PhotoelectricEffect}.

    \subsubsection{Photovoltaischer Effekt}
        Teil des inneren photoelektrischen Effekts, beschreibt Bildung
        eines Photostroms entgegen der Durchlassrichtung des p-n-Übergungs
        als Folge von elektromagnetischer Bestrahlung. Dabei werden
        Ladungsträgerpaare in den dotierten Schichten der des Materials
        getrennt, daworaufhin sich Elektronen in den n-dotierten
        Schichten und die Löcher in den p-dotierten Schichten ablagern
        in so fern sie auf dem weg dahin nicht mit entgegen gesetzen
        Ladungsträgern zusammenstoßen. Der photovolatische Effekt baut
        auf der Photoleitung auf, einem weiterm inneren Teil des
        photoelektrischen Effekts. Der photovoltaische Effekt dient
        als Grundlage für die Funktionsweise von photovoltaischen Zellen
        \cite{Wiki_PhotoelectricEffect}.

    \subsubsection{Photovoltaische Zelle}
        Elektrisches Bauelement das auf Grundlage des photvoltaischen
        Effekts, Strom erzeugt und vorwiegend aus Silizium oder anderen
        Halbleitermaterialien besteht.\\
        Umgangssprachlich werden photovoltaische Zellen oft Solarzellen
        genannt, dabei wird der Begriff auch als Synonym für Solarpanäle
        benutzt obwohl diese eine Ansammlung von photolotaischen Zellen,
        Leiterelementen und strukturellen Bauelementen beschreiben.

\newpage

\subsection{Historie der Photovoltaik}
    \subsubsection{Entdeckung}
        Die Effekte der Photovoltaik wurden erstmals in 1839 von Andre
        Edmond Becquerel entdeckt, aber erst weit später praktisch
        angewendet. Bei der Entdeckung handelte es sich um einen geringen
        Strom zwischen einer Platin-Anode und -Kathode welcher sich unter
        Licht verstärkte. \cite{Wiki_PhotovoltaicHistory}

    \subsubsection{Nennenswerte Ereignisse}
        \begin{itemize}
            \item 1876 - Beweis der direkten Konversion von
                elektromagnetischer Strahlung in elektrische Energie durch
                William Grylls Adams und Richard Evans Day
            \item 1907 - Theoretische Erklärung des photoelektrischen Effekts
                auf Basis der Lichtquantenhypothese (1905) durch Albert
                Einstein
            \item 1912 - 1916 - Experimentelle Bestätigung von Einsteins
                Erklärung durch Robert Andrews Millikan
            \item 1958 - Erste Verwendung von Solarzellen zur Versorgung
                eines Satelliten der NASA (Vanguard I) \cite{Wiki_Vanguard}
        \end{itemize}