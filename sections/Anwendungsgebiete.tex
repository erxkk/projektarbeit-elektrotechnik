\subsection{Niedrigverbrauchapplikationen}
    Solarzellen eignen sich hervorragend zur Energieversorgung für
    vorwiegend bedienungsfreie Applikationen wie eigenständige
    Sensoren (z.B.: Wetterstation, Luft- oder Wasserqualitätssensoren),
    da diese oft einen geringen stetigen Energieverbrauch aufweisen,
    welcher bei Nacht oder schlechten Wetterverhältnissen mit einer
    Batterie überbrückt werden kann.

\subsection{Raumfahrt}
    Auch für die Stromversorgung von Satelliten oder Raumstationen
    eignen siche Solarzellen hervorragend, da im Vakuum das Auffangen
    elektromagnetischer Strahlung nicht durch atmosphärische Effekte
    beeinflusst wird. Die geringe Masse der Zellen und die draus
    resultierenede Treibstoffeffizienz ist ein weiterer Grund für die
    Nutzung in der Raumfahrt.

\subsection{Alternative zu fossilen Brennstoffen}
    \subsubsection{Kalifornien, USA}
        Der US-Bundesstaat Kalifornien beitet ein gutes Besipiel für
        sowohl Vorteile als auch Nachteile von Solarenergie. In
        hinreichend sonnigen Regionen wie Kalifornien reicht die
        durschnittliche durch Solar produzierte Energie zum Decken des
        durschnittlichen Energieverbrauchs aus. Allerdings sind sowohl
        Produktion als auch Verbrauch von Energie nicht konstant.
        Solaranlagen produzieren ihre Energie hauptsächlich zwischen
        7 Uhr und 18 Uhr.
        \begin{figure}[H]
            \centering
            \includegraphics[width=0.9\linewidth]
            {california_supply_2020-08-01.png}
            \caption{Angebot an Energie in Kalifornien, USA, 01.08.2020
                \cite{Img_CaliforniaSupply}
            }
        \end{figure}
        Der tägliche Verbrauch erreicht vorallem um 17 Uhr bis 22 Uhr
        Höchstwerte. Nächte und Schlechtwettertage müssen dann durch
        gespeicherte, importierte oder lokale nicht-Solarenergie überbrückt
        werden, trotz der großartigen Vorraussetzungen für Solarenergie.
        \cite{SolarCalifornia, YouTube_RE-California}
        \begin{figure}[H]
            \centering
            \includegraphics[width=0.9\linewidth]
            {california_demand_2020-08-01.png}
            \caption{Nachfrage an Energie in Kalifornien, USA, 01.08.2020
                \cite{Img_CaliforniaDemand}
            }
        \end{figure}

    \subsubsection{Deutschland}
        In Deutschland wird Solarenergie seit 2000 zunehmend ausgebaut
        und staatlich gefördert, von 2000 bis 2011 stieg der
        Solarenergieanteil von 64GWh auf 19TWh, befindet sich aber mit
        einem Energienteil von gerade mal rund 7\% (stand 2019) noch
        lange nicht auf dem Niveau von Kalifornien
        \cite{Wiki_PhotovoltaicGermany}.
        \begin{figure}[H]
            \centering
            \includegraphics[width=0.9\linewidth]
            {bruttostromerzeugung-in-deutschland.jpg}
            \caption{Anteil der Energieproudktion Deutschland, Stand März
                2019 \cite{Img_GermanySupply}
            }
        \end{figure}