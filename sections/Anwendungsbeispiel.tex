% IoT-Sensor 1..2W betrieb durch solarzelle
% solar zellentyp, größe, kapazität, dunkelheits überbrückung
\subsection{Problemstellung}
    Gegeben sei einen Wettersensor in einem abgelegenen Teil der Sahara,
    welche eine Leistungsaufnahme zwischen 1W und 2W besitzt. Wie kann
    eine solche Anlage kostengünstig und relativ wartungsfrei betrieben
    werden?
    \\\\
    Die Antowrt ist natürlich die starke Sonnenenergie der Region auszunutzen,
    eine Solareinrichtung von geringer Größe kann über den Tag genug
    Sonnenenergie sammeln um sowohl den Wettersensor zu versorgen als auch
    eine Batterie zur Überbrückung der Nacht auf zu laden.
    \\\\
    Um die dauerhafte Reliabilität des Wettersensors zu gewährleisten müssen
    zu jedem Zeitpunkt also 2W an Leistung zur verfügungstehen.
\subsection{Theorie}
    Eine photovaoltaische Zelle wandelt eingehende Leistung \( P_e \) mit
    einem Wirkungsgrad \( \eta \) in ausgehende Leistung \( P_a \) um, die
    eingehende Leistung ist eine Funktion von eingehender Bestrahlungsstärke,
    also Sonnenlesitung pro Bestrahlungsfläche \( E_e = [\frac{P_s}{A_s}] \)
    auf die Fläche \( A \) der photovltaischen Zelle.
        \[ f(P_e) = P_a = \eta P_e \]
    Dabei ist die eingehende Leistung
        \[ P_e = E_e \cdot A\]
    woraus folgt dass
        \[ P_a = \eta \cdot E \cdot A \]
    Von den gegebenen Größen sind sowohl die Sonnenenergie, da diese vom
    Standort des Sensors abhängig ist, als auch die ausgehende Lesitung,
    welche dem maximalen Verbrauch des Sensors entspricht, nicht sonderlich
    beeinflussbar. Es folgt also das für eine ausreichende Leistungsabgabe
    der Zelle die Fläche oder der Wirkungsgrad verändert werden müssen.
    \\\\
    Der Wirkungsgrad \( \eta \) einer photovoltaischen Zelle wird zur
    Laborbedingungen getestet, das bedeutet \( T_{STC} = 25 \)°C
    Umgebungstemperatur, \( 1 \frac{kW}{m^2} \) Bestrahlungsstärke und
    1.5 spektraler Distibution \cite{Wiki_SolarEfficiency}. In der Sahara
    beträgt die durschnittliche Temperatur am Tag rund 30°C und die
    Bestrahlungsstärke rund \( 2 \frac{kW}{m^2} \) bis
    \( 2.5 \frac{kW}{m^2} \) \cite{Wiki_SolarAfrica}. Der Simplizität halber
    werden die folgenden Berechnungen \( E_e = 2.25 \frac{kW}{m^2} \), eines
    Basiswirkungsgrades von \( \eta_{STC} = 0.15 \) und
    Lesitunstemperaturkoeefizenten \( \gamma = -\frac{0.0043}{K} \) aus dem
    Beispieldatenblatt eines Solarmoduls berechnet. \cite{SolarDatasheet}
    \\\\
    Aus den gegebenen Werten ergibt sich die Temperaturdifferenz
        \[ \Delta T = T - T_{STC} = 5 \]
    und der draus resultierende wirkliche Wirkungsgrad bei Betriebstemperatur
    \( T \)
        \[ \eta = \eta_{STC} + \gamma \Delta T = 0.1285\]
    Daraus folgt die benötigte Größe \( A \) der photovoltaischen Zelle in
    Abhängigkeit von eingehender Sonnenenergie, temperaturangepasstem
    Wirkungsgrad und benötigter Ausgangsleistung.
        \[ A = \frac{P_a}{\eta E_e} \approx 0.00691742 m^2 \approx 69.1742 cm^2\]
\subsection{Entwurf}
    % TODO
\subsection{Konklusion}